\documentclass{article}
\usepackage[utf8]{inputenc}
% vim:encoding=utf8:
\usepackage[T1]{fontenc}
%\usepackage[polish]{babel}
\usepackage{frcursive}
\begin{document}
\cursive
\calseries\slshape

I don't really know what this text says, I just found it on a Polish web page
about TeX...

\bigskip

\noindent
Dlaczego używa się TeX-a?
\smallskip

TeX umożliwia efektywne składanie tekstów o dowolnej trudności. Unikalny
algorytm, którym posługuje się TeX przy składaniu akapitów, powoduje, że nie
ma programu oferującego w tym względzie lepsze możliwości. Inne zalety TeX-a
to jego cena i powszechność. System TeX jest oprogramowaniem public domain, co
oznacza, że każdy może zostać jego legalnym użytkownikiem bez żadnych opłat
licencyjnych.

Ponadto TeX, jak każdy prawdziwy program public domain, jest dostępny łącznie
z kodem źródłowym i został zaimplementowany praktycznie na każdej platformie.
W rezultacie użytkownicy TeX-a na całym świecie mogą się porozumiewać (np.
wymieniać dokumenty poprzez pocztę elektroniczną) bez względu na to, na jakim
sprzęcie pracują. TeX działa tak samo na wszystkich platformach.

Wreszcie TeX jest oprogramowaniem otwartym, przez co rozumieć należy jego
zdolność do współpracy z innymi programami. Częstą sytuacją jest wykorzystanie
TeX-a - programu, który doskonale działa w trybie wsadowym - jako ważnego
elementu zautomatyzowanych systemów publikacyjnych, np. opartych na
standardzie SGML. 

\end{document}
